
%%% Local Variables:
%%% mode: latex
%%% TeX-master: "propa"
%%% End:


\section{Design by Contract}
\subsection{Software Korrektheit}
Software ist korrekt, wenn die Spezifikation erfüllt ist.\\
Idee: Spezifiziere Verhalten von Methoden mit Contracts + Teste / Beweise das Verhalten

\subsection{Contract}
Definieren:
\begin{itemize}
  \item P Precondition: Was muss der Client garantieren
  \item C Statements: Was passiert
  \item Q Postcondition: Was kann der Client erwarten
\end{itemize}
Formal stellt dies ein \textit{Hoare Tripel} \(\{P\} C \{Q\}\) der Hoare Logik dar.

\subsection{Java Modeling Language (JML)}
Formale Definition von Interface für Java. Basiert auf Design by Contract. Definiton mit \code{@requires} und
\code{@ensures}.\\
Syntax:
\begin{itemize}
  \item \(a ==> b\) --- a implziert b
  \item \(a <==> b\) --- a iff b
  \item \(a <=!=> b\) --- \(!(a<==>b)\)
  \item \(\backslash result\) --- Ergebnis der Methode
  \item \(\backslash old(E)\) --- Wert von E im Status vor der Methodenausführung
\end{itemize}
Auch möglich \(\backslash forall\) oder \(\backslash exists\) zu verwenden.\\
Ermöglicht außerdem Klassen invarianten anzugeben.\\
Methoden können als \(@pure\) annotiert werden \(\Rightarrow\) keine Seiteneffekte erlaubt.

\subsection{Liskov Substitution Principle}
Für überschreibende Methoden gilt:\\
Precondition darf nicht strenger sein.\\
Postcondition muss mindestens so streng sein.\\

\subsection{Precondition Availability}
Alle Precondition Variablen müssen allen möglichen Callern auch zur Verfügung stehen. Dafür müssen ggf. getter / setter
eingeführt werden.

\subsection{Checking Contracts}
Können statisch / dynamisch oder durch Testcases gecheckt werden.
In Java auch Assertion möglich, allerdings sollten Assertions nicht als Kontrollstruktur verwendet werden.
  
    
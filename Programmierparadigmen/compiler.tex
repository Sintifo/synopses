
%%% Local Variables:
%%% mode: latex
%%% TeX-master: "propa"
%%% End:

\section{Compiler}

\subsection{Ebenen der Übersetzung}
\textbf{Reiner Interpretierer}\\
Lese Quelltext Anweisung für Anweisung + führe aus.\\
Sinnvoll bei Kommandosprachen.\\
z.B. Unix-Shell\\

\textbf{Interpretation nach Vorübersetzung}\\
Transformiere Quellcode in Zwischensprache. Leichter zu interpretieren.\\
z.B. Java-Bytecode, Pascal P-Code, Python\\

\textbf{Vollständige Übersetzung}\\
Übersetze Anwendung in Maschinencode. Code wird zur Laufzeit genutzt.\\
z.B. C/C++, Fortran (Compiler)\\

\textbf{Just-in-time-Compiler}\\
Übersetze zur Laufzeit bei Bedarf. Ist schneller als reine Interpretation.\\ gs
z.B. .NET

\subsection{SIMPLE}
Einfache Programmiersprache zum Verdeutlichen der Konzepte.
Stark eingeschränkte Syntax.\\
Compiler nach Java-Bytecode auf Vorlesungshomepage.\\

Nur Prozeduren, startet mit \code{void main()}. Nur \code{int, boolean}.
\code{int read()} und \code{void println(int value)} in Standardbibliothek.

\subsection{Phasen des Compilers}
\textbf{Lexikalische Analyse}\\
Erkenne bedeutungstragende Zeichengruppen: Tokens.\\
Fasse in Stringtabelle zusammen.\\

\textbf{Syntaktische Analyse}\\
Überprüfe ob in kontextfreier Sprache und erstelle Abstrakten Syntaxbaum.\\

\textbf{Semantische Analyse}\\
Kontextsensitive Analyse nach Deklaration und Verwendung, Typanalyse und Konsistenzprüfung.\\
Erstellt attributierten Syntaxbaum.\\

\textbf{Zwischencodegenerator, Optimierung}\\
Bringt Code in sprach- und zielunabhängige Zwischensprache. Wende Optimierungstechniken an.\\

\textbf{Codegenerierung}\\
Erzeuge Code angepasst auf Zielsystem / Codeauswahl / Scheduling / Register.